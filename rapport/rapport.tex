\documentclass[a4paper,12pt,oneside]{article}
\usepackage[T1]{fontenc}
\usepackage[utf8]{inputenc}
\usepackage{lmodern}
\usepackage[french]{babel}
\usepackage{url,csquotes}
\usepackage[hidelinks,hyperfootnotes=false]{hyperref}
\usepackage[titlepage,fancysections,pagenumber]{polytechnique}
\usepackage{longtable}
\usepackage{listings}
\usepackage{upquote}
\lstset{
	language=C++,
	breaklines=true,
	keywordstyle=\color{blue},
	%numbers=left,
	%numbersep=5pt,
	%numberstyle=\color{green},
	stringstyle=\color{red},
	basicstyle=\ttfamily,
	showstringspaces=false
}

\title{Schur Complement}
\subtitle{INF560 - Algorithmique parallèle et distribuée}
\author{Mario \bsc{Ynocente Castro} \\ Promotion X2013}

\begin{document}

\maketitle

\section{Implementation}

Functions implemented in DirectSolver:
\\
\begin{itemize}
\item
\textbf{LU} : Implementation of LU factorization.
\\
\item
\textbf{Forward} : forward substitution for LU factorization.
\\
\item
\textbf{Backward} : backward substitution for LU factorization.
\\
\item
\textbf{SolveLU} : uses the three previous functions to solve $Ax = rhs$.
\\
\end{itemize}

Functions implemented in Schur:
\\
\begin{itemize}
\item
\textbf{SplitMatrixToBlock} : given list{\_}node{\_}i and list{\_}node{\_}p it divides the given matrix A into four submatrices,
\\
\item
\textbf{CheckTranspose} : given Kip and Kpi it outputs the maximum difference between the corresponding values after transposing one of the matrices.
\\
\item
\textbf{LocalMatrixToGlobalPositions} : given a local matrix, it uses the array l2g to take all the values in this matrix to their original positions in a matrix
of the size of the original matrix.
\\
\item
\textbf{ReconstructK} : reconstructs the original matrix from the local parts and outputs it to the file indicated by gmatrix{\_}filename.
\\
\item
\textbf{SolveSystem} : given the local matrix and the information about the interfaces it calculates x{\_}local using Algorithm 1 and Algorithm 2
as described in the statement.
\end{itemize}

\newpage
\section{Execution}

An example of execution and test of the implementation could be done as follows:

\begin{enumerate}
\item
./DemoLUSolver ../input/cube-125/cube-125 ../output/cube-125/cube-125-lu
\\
\item
mpirun -n 2 DemoSolveSchur ../input/cube-125{\_}2/cube-125 csv ../input/cube-125{\_}2/cube-125{\_}b csv ../output/cube-125{\_}2/cube-125
\\
\item
diff ../output/cube-125/cube-125-lu{\_}x.csv ../output/cube-125{\_}2/cube-125{\_}txg{\_}2{\_}000000.csv
\end{enumerate}


\end{document}
